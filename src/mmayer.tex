\documentclass[10pt,a4paper,ragged2e]{resume}

\geometry{left=2cm,right=10cm,marginparwidth=6.8cm,marginparsep=1.2cm,top=1.25cm,bottom=1.25cm}
\ifxetexorluatex
\setmainfont{Carlito}
\else
\usepackage[utf8]{inputenc}
\usepackage[T1]{fontenc}
\usepackage[default]{lato}
\usepackage{biblatex}
\fi

\definecolor{VividPurple}{HTML}{000000}
\definecolor{SlateGrey}{HTML}{2E2E2E}
\definecolor{LightGrey}{HTML}{2E2E2E}
\colorlet{heading}{VividPurple}
\colorlet{accent}{VividPurple}
\colorlet{emphasis}{SlateGrey}
\colorlet{body}{LightGrey}

\renewcommand{\itemmarker}{{\small\textbullet}}
\renewcommand{\ratingmarker}{\faCircle}

\addbibresource{main.bib}

\begin{document}
    \name{Rusu Dinu-Ștefan}
    \tagline{Software Engineer}
    \personalinfo{
        \email{dinustefan89.ds@gmail.com\hspace{8 mm}}
        \location{Bucharest, Romania}
        \phone{0756478663}

        \linkedin{linkedin.com/in/dinu-stefan-rusu/}
        \github{github.com/rusudinu}
    }

    \begin{fullwidth}
        \makecvheader
    \end{fullwidth}

    \AtBeginEnvironment{itemize}{\small}


    \cvsection[page1sidebar]{Education}
    \cvevent{Bachelor in Computer Engineering}{University Politechnica of Bucharest }{Oct 2020 -- Jun 2024}{}
    \cvevent{Bachelor degree}{"Tudor Vianu" National College of Computer Science }{Sept 2016 -- May 2020}{}


    \cvsection{Experience}
    \cvevent{WEB \& Mobile Software Developer}{Cosmin Savu}{Dec 2019 -- March 2021}{Bucharest, Romania}
    \begin{itemize}
        \item Cosmin Savu is an important investigational journalist in Romania.
        \item Developed his website and cross-platform Mobile Apps in an Agile manner, using React, Flutter, Spring Boot, MongoDB, Firestore, Elasticsearch and a custom-built CDN for fast-size-optimized images.
        \item Hybrid-Cloud architecture, connected cloud services with self-hosted ones.
        \item Optimized for high loads.
        10\% faster compared to other news/blog websites.
        %\smallskip
    \end{itemize}

    \divider

    \cvevent{Android Software Developer}{Rubrika}{Aug 2019 -- Dec 2019}{Bucharest, Romania}
    \begin{itemize}
        \item Developed the Rubrika Android app from scratch, using native code (Java) and Android Studio.
        The Android app is up to 10\% faster than the website.
        \smallskip
    \end{itemize}

    \cvsection{Achievements}
    \smallskip
    \begin{itemize}
        \item Placed 2nd place at the Scientific Session - UPB 2021
        \item Placed 2nd place at "Infoeducatie Nationala" 2020
        \item Placed 3rd place at "Infoeducatie Online" 2020
        \item Placed in top 15\% at "Infoeducatie Nationala" 2019
        \item Participated at \textquotedblleft Performanțe Vianiste\textquotedblright, XI th. Edition, 2019
        \item Participated at \textquotedblleft DaVinci Contest, Reinvent the Future\textquotedblright, IV th edition, 2018
        \item Participated at \textquotedblleft Performanțe Vianiste\textquotedblright, X th. Edition, 2018
    \end{itemize}

    \cvsection{Skills}
    \cvskillnoprogress{Java, Dart, C++, Javascript, C\#, Node.js}
    \smallskip
    \cvskillnoprogress{React, Flutter, Spring Boot}
    \smallskip
    \cvskillnoprogress{Kubernetes, Docker}
    \smallskip
    \cvskillnoprogress{Firebase, Firestore, MongoDB, Elasticsearch, AWS}
    \smallskip
    \cvskillnoprogress{Mobile Development, Linux, Git, Microservices, Agile, Scrum, Hybrid Cloud, NGINX, Grafana, Kibana, Prometheus}
    \smallskip
    \cvskillnoprogress{Illustrator, Lightroom, Premiere Pro}

    \begin{fullwidth}
        \cvsection{Experience \& Other Projects}
        \cvevent{Choice Maker: Smart Decision Making}{}{May 2021}{}
        \begin{itemize}
            \item Developed a cross-platform decision-making app, in Flutter, with Spring Boot backend.
            \item The app helps the user take the best decision possible using a decisional matrix algorithm and regression-based Artificial Intelligence.
            \item Available for Android, IOS, Windows and Web.
        \end{itemize}
        \divider

        \cvevent{PVP Minesweeper}{}{March 2021 - April 2021}{}
        \begin{itemize}
            \item Built a cross-platform mobile app, PVP flavor of the popular Minesweeper game.
            \item Various leaderboards and board skins, selectable in-app.
            \item Used Spring Boot (Websockets, MongoDB) and Flutter.
        \end{itemize}
        \divider

        \cvevent{Anonymoose: Secure Messaging}{}{March 2021 - April 2021}{}
        \begin{itemize}
            \item Developed a cross-platform secure messaging app, in Flutter, with Spring Boot backend.
            \item The app is broadcast-based, and it's using WebSockets.
            \item Available for Android, IOS, Windows and Web.
        \end{itemize}
        \divider

        \cvevent{Website Paul Angelescu}{}{March 2021}{}
        \begin{itemize}
            \item Developed Paul Angelescu's website in an Agile manner, using React, Spring Boot, MongoDB, Firestore and a custom-built CDN for fast-size-optimized images.
            \item Hybrid-Cloud architecture, connected cloud services with self-hosted ones.
            \item Optimized for high loads, 10\% faster compared to other news/blog websites.
            \item Developed in under 1 week.
        \end{itemize}
        \divider

        \cvevent{Enterprise Web Host \& Server Admin}{}{February 2021 - present}{}
        \begin{itemize}
            \item Built and maintained a server used for Web Hosting (microservices \& other backend-specific stuff).
            \item Set up all the good stuff like: RHEL8 (Red Hat Enterprise Linux 8), NGINX (reverse proxy, cache, SSL, HTTP2, backup server redirects), Prometheus, Grafana, MongoDB, Elasticsearch, Kibana, Prometheus, Docker.
            \item Custom fan speed controls using IPMI.
            \item Auto server restart on power loss.
            \item Auto app-restart (for the hosted microservices / apps) on power loss / server crash.
            \item Auto SSL certificate renew.
            \item More than 10 concurrent hosted apps so far.
            \item 24/7 uptime.
        \end{itemize}
        \divider

        \cvevent{subiectebac.ro}{}{December 2020 -- February 2021}{}
        \begin{itemize}
            \item subiectebac.ro is a website developed in an Agile manner, using Spring Boot, and was built to help College Students have an easier and better learning experience for the National Exams.
            \itme The websites provides an archive with all the subjects and solutions for all the previous years.
            \item Minimalist and responsive UI for the best experience both on desktop and mobile.
        \end{itemize}
        \divider

        \cvevent{CRUD Projects Website}{}{December 2020}{}
        \begin{itemize}
            \item Developed an open-source, scalable, CRUD Projects Website using Python 3, Django, Docker, Gunicorn, SQLITE3.
        \end{itemize}

        \cvevent{Subiecte BAC Matematica \& Other related flavors of the app}{}{November 2020 -- December 2020}{}
        \begin{itemize}
            \item Developed a mobile app for College Students that are preparing for the math National Exam. The flavors of the app are: An app with all the materials, and child-apps that contain only profile-specific materials, in order to optimize app size.
            \item The app was built using Flutter, Firebase.
            \item Dark and light modes for a better user experience.
        \end{itemize}
        \divider

        \cvevent{Game Assembler - "Fantasy Game Console"}{}{Hackathon "Infoeducatie" 2020 - 24h}{}
        \begin{itemize}
            \item Developed with my teammate an Assembler, written in C++, for packing games for the emulator of an old console.
        \end{itemize}
        \divider

        \cvevent{Game Emulator - "Fantasy Game Console"}{}{Hackathon "Infoeducatie" 2020 - 24h}{}
        \begin{itemize}
            \item Developed with my teammate an Emulator, written in C++, for playing games packed by the emulator.
            \item The hackathon request for the Emulator was: "RAM: 128 bytes.
            ROM: 4000 bytes.
            Video RAM: 3840 bytes.
            Resolution: 40 x 192 pixels.
            6 general registers.
            4 special registers for the program counter, stack indicator, input data, and a utility of your choice.
            6507 instructions per frame, 60 frames per second.
            All registers are one byte in size.
            Only integer operations are allowed."
        \end{itemize}
        \divider

        \cvevent{Kubernetes Cluster using 3 Raspberry Pi4s}{}{December 2020}{}
        \begin{itemize}
            \item Set up a high performance Kubernetes Cluster using three Raspberry Pi4s, to host auto-scaling containers for the Drivers Data microservices.
        \end{itemize}
        \divider


        \cvevent{Dogs Quiz}{}{April 2020}{}
        \begin{itemize}
            \item Developed a quiz mobile app with dog breeds with the goal to teach kids the different dog breeds.
            \item The app has over 20 thousand dog images, 120 dog breeds, pulled from an open-source Breeds API.
        \end{itemize}
        \divider

        \cvevent{Drivers Data for Windows}{}{April 2017 -- December 2017}{}
        \begin{itemize}
            \item Developed the Drivers Data app for windows, using C\# and a custom built database.
            \item This was the first iteration of Drivers Data.
        \end{itemize}
    \end{fullwidth}


    \clearpage\nocite{*}
\end{document}
